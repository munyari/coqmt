\documentclass[11pt]{report}

\usepackage[utf8x]{inputenc}
\usepackage[T1]{fontenc}
\usepackage{fullpage}
\usepackage[color]{../../coqdoc}

\input{../common/version}
%%%%%%%%%%%%%%%%%%%%%%%%%%%%%%%%
% File title.tex
% Page formatting commands
% Macro \coverpage
%%%%%%%%%%%%%%%%%%%%%%%%%%%%%%%%

%\setlength{\marginparwidth}{0pt}
%\setlength{\oddsidemargin}{0pt}
%\setlength{\evensidemargin}{0pt}
%\setlength{\marginparsep}{0pt}
%\setlength{\topmargin}{0pt}
%\setlength{\textwidth}{16.9cm}
%\setlength{\textheight}{22cm}
%\usepackage{fullpage}

%\newcommand{\printingdate}{\today}
%\newcommand{\isdraft}{\Large\bf\today\\[20pt]}
%\newcommand{\isdraft}{\vspace{20pt}}

\newcommand{\coverpage}[3]{
\thispagestyle{empty}
\begin{center}
\bfseries % for the rest of this page, until \end{center}
\Huge
The Coq Proof Assistant\\[12pt]
#1\\[20pt]
\Large\today\\[20pt]
Version \coqversion%\footnote[1]{This research was partly supported by IST working group ``Types''}

\vspace{0pt plus .5fill}
#2
\par\vfill
The Coq Development Team

\vspace*{15pt}
\end{center}
\newpage

\thispagestyle{empty}
\hbox{}\vfill % without \hbox \vfill does not work at the top of the page
\begin{flushleft}
%BEGIN LATEX
V\coqversion, \today
\par\vspace{20pt}
%END LATEX
\copyright INRIA 1999-2004 ({\Coq} versions 7.x)

\copyright INRIA 2004-2010 ({\Coq} versions 8.x)

#3
\end{flushleft}
} % end of \coverpage definition


% \newcommand{\shorttitle}[1]{
% \begin{center}
% \begin{huge}
% \begin{bf}
% The Coq Proof Assistant\\
% \vspace{10pt}
%     #1\\
% \end{bf}
% \end{huge}
% \end{center}
% \vspace{5pt}
% }

% Local Variables: 
% mode: LaTeX
% TeX-master: ""
% End: 

% $Id: title.tex 11782 2009-01-13 21:50:05Z herbelin $ 


% macros for coq.tex

\newcommand{\Coq}{\textsf{Coq}}
\newcommand{\CCI}{Calculus of Inductive Constructions}

\newcommand{\refsec}[1]{\textbf{\ref{#1}}}

\begin{document}
 
\coverpage{The standard library}%
{\ }
{This material is distributed under the terms of the GNU Lesser
General Public License Version 2.1.}

\tableofcontents

\newpage
% \section*{The \Coq\ standard library}

This document is a short description of the \Coq\ standard library.
This library comes with the system as a complement of the core library
(the {\bf Init} library ; see the Reference Manual for a description
of this library). It provides a set of modules directly available
through the \verb!Require! command.

The standard library is composed of the following subdirectories:
\begin{description}
  \item[Logic]  Classical logic and dependent equality
  \item[Bool]   Booleans (basic functions and results)
  \item[Arith]  Basic Peano arithmetic
  \item[ZArith] Basic integer arithmetic
  \item[Reals]  Classical Real Numbers and Analysis
  \item[Lists]  Monomorphic and polymorphic lists (basic functions and
                  results), Streams (infinite sequences defined 
                  with co-inductive types)
  \item[Sets]   Sets (classical, constructive, finite, infinite, power set,
                  etc.)
  \item[Relations] Relations (definitions and basic results).
  \item[Sorting] Sorted list (basic definitions and heapsort
                 correctness). 
  \item[Wellfounded] Well-founded relations (basic results).
  \item[Program] Tactics to deal with dependently-typed programs and
    their proofs.
  \item[Classes] Standard type class instances on relations and
    Coq part of the setoid rewriting tactic.
\end{description}


Each of these subdirectories contains a set of modules, whose
specifications (\gallina{} files) have
been roughly, and automatically, pasted in the following pages. There
is also a version of this document in HTML format on the WWW, which
you can access from the \Coq\ home page at
\texttt{http://coq.inria.fr/library}.

\documentclass[11pt]{report}

\usepackage[latin1]{inputenc}
\usepackage[T1]{fontenc}
\usepackage{fullpage}
\usepackage[color]{../../coqdoc}

\input{../common/version}
%%%%%%%%%%%%%%%%%%%%%%%%%%%%%%%%
% File title.tex
% Page formatting commands
% Macro \coverpage
%%%%%%%%%%%%%%%%%%%%%%%%%%%%%%%%

%\setlength{\marginparwidth}{0pt}
%\setlength{\oddsidemargin}{0pt}
%\setlength{\evensidemargin}{0pt}
%\setlength{\marginparsep}{0pt}
%\setlength{\topmargin}{0pt}
%\setlength{\textwidth}{16.9cm}
%\setlength{\textheight}{22cm}
%\usepackage{fullpage}

%\newcommand{\printingdate}{\today}
%\newcommand{\isdraft}{\Large\bf\today\\[20pt]}
%\newcommand{\isdraft}{\vspace{20pt}}

\newcommand{\coverpage}[3]{
\thispagestyle{empty}
\begin{center}
\bfseries % for the rest of this page, until \end{center}
\Huge
The Coq Proof Assistant\\[12pt]
#1\\[20pt]
\Large\today\\[20pt]
Version \coqversion%\footnote[1]{This research was partly supported by IST working group ``Types''}

\vspace{0pt plus .5fill}
#2
\par\vfill
The Coq Development Team

\vspace*{15pt}
\end{center}
\newpage

\thispagestyle{empty}
\hbox{}\vfill % without \hbox \vfill does not work at the top of the page
\begin{flushleft}
%BEGIN LATEX
V\coqversion, \today
\par\vspace{20pt}
%END LATEX
\copyright INRIA 1999-2004 ({\Coq} versions 7.x)

\copyright INRIA 2004-2010 ({\Coq} versions 8.x)

#3
\end{flushleft}
} % end of \coverpage definition


% \newcommand{\shorttitle}[1]{
% \begin{center}
% \begin{huge}
% \begin{bf}
% The Coq Proof Assistant\\
% \vspace{10pt}
%     #1\\
% \end{bf}
% \end{huge}
% \end{center}
% \vspace{5pt}
% }

% Local Variables: 
% mode: LaTeX
% TeX-master: ""
% End: 

% $Id: title.tex 11782 2009-01-13 21:50:05Z herbelin $ 


% macros for coq.tex

\newcommand{\Coq}{\textsf{Coq}}
\newcommand{\CCI}{Calculus of Inductive Constructions}

\newcommand{\refsec}[1]{\textbf{\ref{#1}}}

\begin{document}
 
\coverpage{The standard library}%
{\ }
{This material is distributed under the terms of the GNU Lesser
General Public License Version 2.1.}

\tableofcontents

\newpage
% \section*{The \Coq\ standard library}

This document is a short description of the \Coq\ standard library.
This library comes with the system as a complement of the core library
(the {\bf Init} library ; see the Reference Manual for a description
of this library). It provides a set of modules directly available
through the \verb!Require! command.

The standard library is composed of the following subdirectories:
\begin{description}
  \item[Logic]  Classical logic and dependent equality
  \item[Bool]   Booleans (basic functions and results)
  \item[Arith]  Basic Peano arithmetic
  \item[ZArith] Basic integer arithmetic
  \item[Reals]  Classical Real Numbers and Analysis
  \item[Lists]  Monomorphic and polymorphic lists (basic functions and
                  results), Streams (infinite sequences defined 
                  with co-inductive types)
  \item[Sets]   Sets (classical, constructive, finite, infinite, power set,
                  etc.)
  \item[Relations] Relations (definitions and basic results).
  \item[Sorting] Sorted list (basic definitions and heapsort
                 correctness). 
  \item[Wellfounded] Well-founded relations (basic results).
  \item[Program] Tactics to deal with dependently-typed programs and
    their proofs.
  \item[Classes] Standard type class instances on relations and
    Coq part of the setoid rewriting tactic.
\end{description}


Each of these subdirectories contains a set of modules, whose
specifications (\gallina{} files) have
been roughly, and automatically, pasted in the following pages. There
is also a version of this document in HTML format on the WWW, which
you can access from the \Coq\ home page at
\texttt{http://coq.inria.fr/library}.

\documentclass[11pt]{report}

\usepackage[latin1]{inputenc}
\usepackage[T1]{fontenc}
\usepackage{fullpage}
\usepackage[color]{../../coqdoc}

\input{../common/version}
%%%%%%%%%%%%%%%%%%%%%%%%%%%%%%%%
% File title.tex
% Page formatting commands
% Macro \coverpage
%%%%%%%%%%%%%%%%%%%%%%%%%%%%%%%%

%\setlength{\marginparwidth}{0pt}
%\setlength{\oddsidemargin}{0pt}
%\setlength{\evensidemargin}{0pt}
%\setlength{\marginparsep}{0pt}
%\setlength{\topmargin}{0pt}
%\setlength{\textwidth}{16.9cm}
%\setlength{\textheight}{22cm}
%\usepackage{fullpage}

%\newcommand{\printingdate}{\today}
%\newcommand{\isdraft}{\Large\bf\today\\[20pt]}
%\newcommand{\isdraft}{\vspace{20pt}}

\newcommand{\coverpage}[3]{
\thispagestyle{empty}
\begin{center}
\bfseries % for the rest of this page, until \end{center}
\Huge
The Coq Proof Assistant\\[12pt]
#1\\[20pt]
\Large\today\\[20pt]
Version \coqversion%\footnote[1]{This research was partly supported by IST working group ``Types''}

\vspace{0pt plus .5fill}
#2
\par\vfill
The Coq Development Team

\vspace*{15pt}
\end{center}
\newpage

\thispagestyle{empty}
\hbox{}\vfill % without \hbox \vfill does not work at the top of the page
\begin{flushleft}
%BEGIN LATEX
V\coqversion, \today
\par\vspace{20pt}
%END LATEX
\copyright INRIA 1999-2004 ({\Coq} versions 7.x)

\copyright INRIA 2004-2010 ({\Coq} versions 8.x)

#3
\end{flushleft}
} % end of \coverpage definition


% \newcommand{\shorttitle}[1]{
% \begin{center}
% \begin{huge}
% \begin{bf}
% The Coq Proof Assistant\\
% \vspace{10pt}
%     #1\\
% \end{bf}
% \end{huge}
% \end{center}
% \vspace{5pt}
% }

% Local Variables: 
% mode: LaTeX
% TeX-master: ""
% End: 

% $Id: title.tex 11782 2009-01-13 21:50:05Z herbelin $ 


% macros for coq.tex

\newcommand{\Coq}{\textsf{Coq}}
\newcommand{\CCI}{Calculus of Inductive Constructions}

\newcommand{\refsec}[1]{\textbf{\ref{#1}}}

\begin{document}
 
\coverpage{The standard library}%
{\ }
{This material is distributed under the terms of the GNU Lesser
General Public License Version 2.1.}

\tableofcontents

\newpage
% \section*{The \Coq\ standard library}

This document is a short description of the \Coq\ standard library.
This library comes with the system as a complement of the core library
(the {\bf Init} library ; see the Reference Manual for a description
of this library). It provides a set of modules directly available
through the \verb!Require! command.

The standard library is composed of the following subdirectories:
\begin{description}
  \item[Logic]  Classical logic and dependent equality
  \item[Bool]   Booleans (basic functions and results)
  \item[Arith]  Basic Peano arithmetic
  \item[ZArith] Basic integer arithmetic
  \item[Reals]  Classical Real Numbers and Analysis
  \item[Lists]  Monomorphic and polymorphic lists (basic functions and
                  results), Streams (infinite sequences defined 
                  with co-inductive types)
  \item[Sets]   Sets (classical, constructive, finite, infinite, power set,
                  etc.)
  \item[Relations] Relations (definitions and basic results).
  \item[Sorting] Sorted list (basic definitions and heapsort
                 correctness). 
  \item[Wellfounded] Well-founded relations (basic results).
  \item[Program] Tactics to deal with dependently-typed programs and
    their proofs.
  \item[Classes] Standard type class instances on relations and
    Coq part of the setoid rewriting tactic.
\end{description}


Each of these subdirectories contains a set of modules, whose
specifications (\gallina{} files) have
been roughly, and automatically, pasted in the following pages. There
is also a version of this document in HTML format on the WWW, which
you can access from the \Coq\ home page at
\texttt{http://coq.inria.fr/library}.

\documentclass[11pt]{report}

\usepackage[latin1]{inputenc}
\usepackage[T1]{fontenc}
\usepackage{fullpage}
\usepackage[color]{../../coqdoc}

\input{../common/version}
\input{../common/title}
\input{../common/macros}

\begin{document}
 
\coverpage{The standard library}%
{\ }
{This material is distributed under the terms of the GNU Lesser
General Public License Version 2.1.}

\tableofcontents

\newpage
% \section*{The \Coq\ standard library}

This document is a short description of the \Coq\ standard library.
This library comes with the system as a complement of the core library
(the {\bf Init} library ; see the Reference Manual for a description
of this library). It provides a set of modules directly available
through the \verb!Require! command.

The standard library is composed of the following subdirectories:
\begin{description}
  \item[Logic]  Classical logic and dependent equality
  \item[Bool]   Booleans (basic functions and results)
  \item[Arith]  Basic Peano arithmetic
  \item[ZArith] Basic integer arithmetic
  \item[Reals]  Classical Real Numbers and Analysis
  \item[Lists]  Monomorphic and polymorphic lists (basic functions and
                  results), Streams (infinite sequences defined 
                  with co-inductive types)
  \item[Sets]   Sets (classical, constructive, finite, infinite, power set,
                  etc.)
  \item[Relations] Relations (definitions and basic results).
  \item[Sorting] Sorted list (basic definitions and heapsort
                 correctness). 
  \item[Wellfounded] Well-founded relations (basic results).
  \item[Program] Tactics to deal with dependently-typed programs and
    their proofs.
  \item[Classes] Standard type class instances on relations and
    Coq part of the setoid rewriting tactic.
\end{description}


Each of these subdirectories contains a set of modules, whose
specifications (\gallina{} files) have
been roughly, and automatically, pasted in the following pages. There
is also a version of this document in HTML format on the WWW, which
you can access from the \Coq\ home page at
\texttt{http://coq.inria.fr/library}.

\input{Library.coqdoc}

\end{document}

% $Id: Library.tex 11576 2008-11-10 19:13:15Z msozeau $ 


\end{document}

% $Id: Library.tex 11576 2008-11-10 19:13:15Z msozeau $ 


\end{document}

% $Id: Library.tex 11576 2008-11-10 19:13:15Z msozeau $ 


\end{document}

% $Id: Library.tex 12363 2009-09-28 15:04:07Z letouzey $ 
