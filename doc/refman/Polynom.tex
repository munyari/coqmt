\achapter{The \texttt{ring} and \texttt{field} tactic families}
\aauthor{Bruno Barras, Benjamin Gr\'egoire, Assia
  Mahboubi, Laurent Th\'ery\footnote{based on previous work from
  Patrick Loiseleur and Samuel Boutin}}
\label{ring}
\tacindex{ring}

This chapter presents the tactics dedicated to deal with ring and
field equations.

\asection{What does this tactic do?}

\texttt{ring} does associative-commutative rewriting in ring and semi-ring
structures. Assume you have two binary functions $\oplus$ and $\otimes$
that are associative and commutative, with $\oplus$ distributive on
$\otimes$, and two constants 0 and 1 that are unities for $\oplus$ and
$\otimes$. A \textit{polynomial} is an expression built on variables $V_0, V_1,
\dots$ and constants by application of $\oplus$ and $\otimes$.

Let an {\it ordered product} be a product of variables $V_{i_1}
\otimes \ldots \otimes V_{i_n}$ verifying $i_1 \le i_2 \le \dots \le
i_n$. Let a \textit{monomial} be the product of a constant and an
ordered product.  We can order the monomials by the lexicographic
order on products of variables. Let a \textit{canonical sum} be an
ordered sum of monomials that are all different, i.e. each monomial in
the sum is strictly less than the following monomial according to the
lexicographic order. It is an easy theorem to show that every
polynomial is equivalent (modulo the ring properties) to exactly one
canonical sum. This canonical sum is called the \textit{normal form}
of the polynomial. In fact, the actual representation shares monomials
with same prefixes. So what does \texttt{ring}? It normalizes
polynomials over any ring or semi-ring structure. The basic use of
\texttt{ring} is to simplify ring expressions, so that the user does
not have to deal manually with the theorems of associativity and
commutativity.

\begin{Examples}
\item In the ring of integers, the normal form of 
$x (3 + yx + 25(1 - z)) + zx$ is $28x + (-24)xz + xxy$.
\item For the classical propositional calculus (or the boolean rings)
  the normal form is what logicians call \textit{disjunctive normal
    form}: every formula is equivalent to a disjunction of
  conjunctions of atoms. (Here $\oplus$ is $\vee$, $\otimes$ is
  $\wedge$, variables are atoms and the only constants are T and F)
\end{Examples}

\texttt{ring} is also able to compute a normal form modulo monomial 
equalities. For example, under the hypothesis that $2x^2 = yz+1$,
 the normal form of $2(x + 1)x - x - zy$ is $x+1$.

\asection{The variables map}

It is frequent to have an expression built with + and
  $\times$, but rarely on variables only.
Let us associate a number to each subterm of a ring
expression in the \gallina\ language. For example in the ring
\texttt{nat}, consider the expression:

\begin{quotation}
\begin{verbatim}
(plus (mult (plus (f (5)) x) x)
      (mult (if b then (4) else (f (3))) (2)))
\end{verbatim}
\end{quotation}

\noindent As a ring expression, it has 3 subterms. Give each subterm a
number in an arbitrary order:

\begin{tabular}{ccl}
0 & $\mapsto$ & \verb|if b then (4) else (f (3))| \\
1 & $\mapsto$ & \verb|(f (5))| \\
2 & $\mapsto$ & \verb|x| \\
\end{tabular}

\noindent Then normalize the ``abstract'' polynomial 

$$((V_1 \otimes V_2) \oplus V_2) \oplus (V_0 \otimes 2) $$

\noindent In our example the normal form is:

$$(2 \otimes V_0) \oplus (V_1 \otimes V_2) \oplus (V_2 \otimes V_2)$$

\noindent Then substitute the variables by their values in the variables map to
get the concrete normal polynomial:

\begin{quotation}
\begin{verbatim}
(plus (mult (2) (if b then (4) else (f (3)))) 
      (plus (mult (f (5)) x) (mult x x))) 
\end{verbatim}
\end{quotation}

\asection{Is it automatic?}

Yes, building the variables map and doing the substitution after
normalizing is automatically done by the tactic. So you can just forget
this paragraph and use the tactic according to your intuition.

\asection{Concrete usage in \Coq
\tacindex{ring}
\tacindex{ring\_simplify}}

The {\tt ring} tactic solves equations upon polynomial expressions of
a ring (or semi-ring) structure. It proceeds by normalizing both hand
sides of the equation (w.r.t. associativity, commutativity and
distributivity, constant propagation, rewriting of monomials) 
and comparing syntactically the results.

{\tt ring\_simplify} applies the normalization procedure described
above to the terms given. The tactic then replaces all occurrences of
the terms given in the conclusion of the goal by their normal
forms. If no term is given, then the conclusion should be an equation
and both hand sides are normalized. 
The tactic can also be applied in a hypothesis.

The tactic must be loaded by \texttt{Require Import Ring}. The ring
structures must be declared with the \texttt{Add Ring} command (see
below). The ring of booleans is predefined; if one wants to use the
tactic on \texttt{nat} one must first require the module
\texttt{ArithRing} (exported by \texttt{Arith});
for \texttt{Z}, do \texttt{Require Import
ZArithRing} or simply \texttt{Require Import ZArith}; 
for \texttt{N}, do \texttt{Require Import NArithRing} or 
\texttt{Require Import NArith}.

\Example
\begin{coq_eval}
Reset Initial.
\end{coq_eval}
\begin{coq_example}
Require Import ZArith.
Open Scope Z_scope.
Goal forall a b c:Z,
  (a + b + c)^2  =
  a * a + b^2 + c * c + 2 * a * b + 2 * a * c + 2 * b * c.
\end{coq_example}
\begin{coq_example}
intros; ring.
\end{coq_example}
\begin{coq_eval}
Abort.
\end{coq_eval}
\begin{coq_example}
Goal forall a b:Z, 2*a*b = 30 ->
        (a+b)^2 = a^2 + b^2 + 30.
\end{coq_example}
\begin{coq_example}
intros a b H; ring [H].
\end{coq_example} 
\begin{coq_eval}
Reset Initial.
\end{coq_eval}

\begin{Variants}
  \item {\tt ring [\term$_1$ {\ldots} \term$_n$]} decides the equality of two
    terms modulo ring operations and rewriting of the equalities
    defined by \term$_1$ {\ldots} \term$_n$. Each of \term$_1$
    {\ldots} \term$_n$ has to be a proof of some equality $m = p$,
    where $m$ is a monomial (after ``abstraction''),
    $p$ a polynomial and $=$ the corresponding equality of the ring structure.

  \item {\tt ring\_simplify [\term$_1$ {\ldots} \term$_n$] $t_1 \ldots t_m$ in 
{\ident}}
     performs the simplification in the hypothesis named {\tt ident}.
\end{Variants}

\Warning \texttt{ring\_simplify \term$_1$; ring\_simplify \term$_2$} is
not equivalent to \texttt{ring\_simplify \term$_1$ \term$_2$}. In the
latter case the variables map is shared between the two terms, and
common subterm $t$ of \term$_1$ and \term$_2$ will have the same
associated variable number. So the first alternative should be
avoided for terms belonging to the same ring theory.


\begin{ErrMsgs}
\item \errindex{not a valid ring equation}
  The conclusion of the goal is not provable in the corresponding ring
  theory.
\item \errindex{arguments of ring\_simplify do not have all the same type}
  {\tt ring\_simplify} cannot simplify terms of several rings at the
  same time. Invoke the tactic once per ring structure.
\item \errindex{cannot find a declared ring structure over {\tt term}}
  No ring has been declared for the type of the terms to be
  simplified. Use {\tt Add Ring} first.
\item \errindex{cannot find a declared ring structure for equality
  {\tt term}}
  Same as above is the case of the {\tt ring} tactic.
\end{ErrMsgs}

\asection{Adding a ring structure
\comindex{Add Ring}}

Declaring a new ring consists in proving that a ring signature (a
carrier set, an equality, and ring operations: {\tt
Ring\_theory.ring\_theory} and {\tt Ring\_theory.semi\_ring\_theory})
satisfies the ring axioms. Semi-rings (rings without $+$ inverse) are
also supported. The equality can be either Leibniz equality, or any
relation declared as a setoid (see~\ref{setoidtactics}). The definition
of ring and semi-rings (see module {\tt Ring\_theory}) is:
\begin{verbatim}
 Record ring_theory : Prop := mk_rt {
    Radd_0_l    : forall x, 0 + x == x;
    Radd_sym    : forall x y, x + y == y + x;
    Radd_assoc  : forall x y z, x + (y + z) == (x + y) + z;
    Rmul_1_l    : forall x, 1 * x == x;
    Rmul_sym    : forall x y, x * y == y * x;
    Rmul_assoc  : forall x y z, x * (y * z) == (x * y) * z;
    Rdistr_l    : forall x y z, (x + y) * z == (x * z) + (y * z);
    Rsub_def    : forall x y, x - y == x + -y;
    Ropp_def    : forall x, x + (- x) == 0
 }.

Record semi_ring_theory : Prop := mk_srt {
    SRadd_0_l   : forall n, 0 + n == n;
    SRadd_sym   : forall n m, n + m == m + n ;
    SRadd_assoc : forall n m p, n + (m + p) == (n + m) + p;
    SRmul_1_l   : forall n, 1*n == n;
    SRmul_0_l   : forall n, 0*n == 0; 
    SRmul_sym   : forall n m, n*m == m*n;
    SRmul_assoc : forall n m p, n*(m*p) == (n*m)*p;
    SRdistr_l   : forall n m p, (n + m)*p == n*p + m*p
  }.
\end{verbatim}

This implementation of {\tt ring} also features a notion of constant
that can be parameterized. This can be used to improve the handling of
closed expressions when operations are effective. It consists in
introducing a type of \emph{coefficients} and an implementation of the
ring operations, and a morphism from the coefficient type to the ring
carrier type. The morphism needs not be injective, nor surjective.  As
an example, one can consider the real numbers. The set of coefficients
could be the rational numbers, upon which the ring operations can be
implemented. The fact that there exists a morphism is defined by the
following properties:
\begin{verbatim}
 Record ring_morph : Prop := mkmorph {
    morph0    : [cO] == 0;
    morph1    : [cI] == 1;
    morph_add : forall x y, [x +! y] == [x]+[y];
    morph_sub : forall x y, [x -! y] == [x]-[y];
    morph_mul : forall x y, [x *! y] == [x]*[y];
    morph_opp : forall x, [-!x] == -[x];
    morph_eq  : forall x y, x?=!y = true -> [x] == [y] 
  }.

 Record semi_morph : Prop := mkRmorph {
    Smorph0 : [cO] == 0;
    Smorph1 : [cI] == 1;
    Smorph_add : forall x y, [x +! y] == [x]+[y];
    Smorph_mul : forall x y, [x *! y] == [x]*[y];
    Smorph_eq  : forall x y, x?=!y = true -> [x] == [y] 
  }.
\end{verbatim}
where {\tt c0} and {\tt cI} denote the 0 and 1 of the coefficient set,
{\tt +!}, {\tt *!}, {\tt -!} are the implementations of the ring
operations, {\tt ==} is the equality of the coefficients, {\tt ?+!} is
an implementation of this equality, and {\tt [x]} is a notation for
the image of {\tt x} by the ring morphism.

Since {\tt Z} is an initial ring (and {\tt N} is an initial
semi-ring), it can always be considered as a set of
coefficients. There are basically three kinds of (semi-)rings:
\begin{description}
\item[abstract rings] to be used when operations are not
  effective. The set of coefficients is {\tt Z} (or {\tt N} for
  semi-rings).
\item[computational rings] to be used when operations are
  effective. The set of coefficients is the ring itself. The user only
  has to provide an implementation for the equality.
\item[customized ring] for other cases. The user has to provide the
  coefficient set and the morphism.
\end{description}

This implementation of ring can also recognize simple 
power expressions as ring expressions. A power function is specified by 
the following property:
\begin{verbatim}
 Section POWER.
  Variable Cpow : Set.
  Variable Cp_phi : N -> Cpow.
  Variable rpow : R -> Cpow -> R. 
  
  Record power_theory : Prop := mkpow_th {
    rpow_pow_N : forall r n, req (rpow r (Cp_phi n)) (pow_N rI rmul r n)
  }.

 End POWER.
\end{verbatim}


The syntax for adding a new ring is {\tt Add Ring $name$ : $ring$
($mod_1$,\dots,$mod_2$)}.  The name is not relevent. It is just used
for error messages. The term $ring$ is a proof that the ring signature
satisfies the (semi-)ring axioms. The optional list of modifiers is
used to tailor the behavior of the tactic. The following list
describes their syntax and effects:
\begin{description}
\item[abstract] declares the ring as abstract. This is the default.
\item[decidable \term] declares the ring as computational. The expression 
  \term{} is
  the correctness proof of an equality test {\tt ?=!} (which should be
  evaluable). Its type should be of
  the form {\tt forall x y, x?=!y = true $\rightarrow$ x == y}.
\item[morphism \term] declares the ring as a customized one. The expression 
  \term{} is
  a proof that there exists a morphism between a set of coefficient
  and the ring carrier (see {\tt Ring\_theory.ring\_morph} and {\tt
  Ring\_theory.semi\_morph}).
\item[setoid \term$_1$ \term$_2$] forces the use of given setoid. The 
  expression \term$_1$ is a proof that the equality is indeed a setoid
  (see {\tt Setoid.Setoid\_Theory}), and \term$_2$ a proof that the
  ring operations are morphisms (see {\tt Ring\_theory.ring\_eq\_ext} and
  {\tt Ring\_theory.sring\_eq\_ext}). This modifier needs not be used if the
  setoid and morphisms have been declared.
\item[constants [\ltac]] specifies a tactic expression that, given a term,
  returns either an object of the coefficient set that is mapped to
  the expression via the morphism, or returns {\tt
  InitialRing.NotConstant}. The default behaviour is to map only 0 and
  1 to their counterpart in the coefficient set. This is generally not
  desirable for non trivial computational rings.
\item[preprocess [\ltac]]
  specifies a tactic that is applied as a preliminary step for {\tt
  ring} and {\tt ring\_simplify}. It can be used to transform a goal
  so that it is better recognized. For instance, {\tt S n} can be
  changed to {\tt plus 1 n}.
\item[postprocess [\ltac]] specifies a tactic that is applied as a final step
  for {\tt ring\_simplify}. For instance, it can be used to undo
  modifications of the preprocessor.
\item[power\_tac {\term} [\ltac]] allows {\tt ring} and {\tt ring\_simplify} to
  recognize power expressions with a constant positive integer exponent 
  (example: $x^2$). The term {\term} is a proof that a given power function
  satisfies the specification of a power function ({\term} has to be a
  proof of {\tt Ring\_theory.power\_theory}) and {\ltac} specifies a
  tactic expression that, given a term, ``abstracts'' it into an
  object of type {\tt N} whose interpretation via {\tt Cp\_phi} (the
  evaluation function of power coefficient) is the original term, or
  returns {\tt InitialRing.NotConstant} if not a constant coefficient
  (i.e. {\ltac} is the inverse function of {\tt Cp\_phi}).
  See files {\tt contrib/setoid\_ring/ZArithRing.v} and
  {\tt contrib/setoid\_ring/RealField.v} for examples.
  By default the tactic does not recognize power expressions as ring
  expressions.
\item[sign {\term}] allows {\tt ring\_simplify} to use a minus operation
  when outputing its normal form, i.e writing $x - y$ instead of $x + (-y)$. 
  The term {\term} is a proof that a given sign function indicates expressions
   that are signed ({\term} has to be a
  proof of {\tt Ring\_theory.get\_sign}). See  {\tt contrib/setoid\_ring/IntialRing.v} for examples of sign function.
\item[div {\term}] allows  {\tt ring} and {\tt ring\_simplify} to use moniomals
with coefficient other than 1 in the rewriting. The term {\term} is a proof that a given division function  satisfies the specification of an euclidean
  division function  ({\term} has to be a
  proof of {\tt Ring\_theory.div\_theory}). For example, this function is
  called when trying to rewrite $7x$ by $2x = z$ to tell that $7 = 3 * 2 + 1$.
   See  {\tt contrib/setoid\_ring/IntialRing.v} for examples of div function.
  
\end{description}


\begin{ErrMsgs}
\item \errindex{bad ring structure}
  The proof of the ring structure provided is not of the expected type.
\item \errindex{bad lemma for decidability of equality}
  The equality function provided in the case of a computational ring
  has not the expected type.
\item \errindex{ring {\it operation} should be declared as a morphism}
  A setoid associated to the carrier of the ring structure as been
  found, but the ring operation should be declared as
  morphism. See~\ref{setoidtactics}.
\end{ErrMsgs}

\asection{How does it work?}

The code of \texttt{ring} is a good example of tactic written using
\textit{reflection}.  What is reflection? Basically, it is writing
\Coq{} tactics in \Coq, rather than in \ocaml. From the philosophical
point of view, it is using the ability of the Calculus of
Constructions to speak and reason about itself.  For the \texttt{ring}
tactic we used \Coq\ as a programming language and also as a proof
environment to build a tactic and to prove it correctness.

The interested reader is strongly advised to have a look at the file
\texttt{Ring\_polynom.v}. Here a type for polynomials is defined: 

\begin{small}
\begin{flushleft}
\begin{verbatim}
Inductive PExpr : Type :=
  | PEc : C -> PExpr
  | PEX : positive -> PExpr
  | PEadd : PExpr -> PExpr -> PExpr
  | PEsub : PExpr -> PExpr -> PExpr
  | PEmul : PExpr -> PExpr -> PExpr
  | PEopp : PExpr -> PExpr
  | PEpow : PExpr -> N -> PExpr.
\end{verbatim}
\end{flushleft}
\end{small}

Polynomials in normal form are defined as:
\begin{small}
\begin{flushleft}
\begin{verbatim}
 Inductive Pol : Type :=
  | Pc : C -> Pol 
  | Pinj : positive -> Pol -> Pol                   
  | PX : Pol -> positive -> Pol -> Pol.
\end{verbatim}
\end{flushleft}
\end{small}
where {\tt Pinj n P} denotes $P$ in which $V_i$ is replaced by
$V_{i+n}$, and {\tt PX P n Q} denotes $P \otimes V_1^{n} \oplus Q'$,
$Q'$ being $Q$ where $V_i$ is replaced by $V_{i+1}$. 


Variables maps are represented by list of ring elements, and two
interpretation functions, one that maps a variables map and a
polynomial to an element of the concrete ring, and the second one that
does the same for normal forms:
\begin{small}
\begin{flushleft}
\begin{verbatim}
Definition PEeval : list R -> PExpr -> R := [...].
Definition Pphi_dev : list R -> Pol -> R := [...].
\end{verbatim}
\end{flushleft}
\end{small}

A function to normalize polynomials is defined, and the big theorem is
its correctness w.r.t interpretation, that is:

\begin{small}
\begin{flushleft}
\begin{verbatim}
Definition norm : PExpr -> Pol := [...].
Lemma Pphi_dev_ok :
   forall l pe npe, norm pe = npe -> PEeval l pe == Pphi_dev l npe.
\end{verbatim}
\end{flushleft}
\end{small}

So now, what is the scheme for a normalization proof? Let \texttt{p}
be the polynomial expression that the user wants to normalize. First a
little piece of ML code guesses the type of \texttt{p}, the ring
theory \texttt{T} to use, an abstract polynomial \texttt{ap} and a
variables map \texttt{v} such that \texttt{p} is
$\beta\delta\iota$-equivalent to \verb|(PEeval v ap)|. Then we
replace it by \verb|(Pphi_dev v (norm ap))|, using the
main correctness theorem and we reduce it to a concrete expression
\texttt{p'}, which is the concrete normal form of
\texttt{p}. This is summarized in this diagram:
\begin{center}
\begin{tabular}{rcl}
\texttt{p} & $\rightarrow_{\beta\delta\iota}$  
   & \texttt{(PEeval v ap)} \\
 & & $=_{\mathrm{(by\ the\ main\ correctness\ theorem)}}$ \\
\texttt{p'} 
   & $\leftarrow_{\beta\delta\iota}$ 
   & \texttt{(Pphi\_dev v (norm ap))}
\end{tabular}
\end{center}
The user do not see the right part of the diagram. 
From outside, the tactic behaves like a
$\beta\delta\iota$ simplification extended with AC rewriting rules.
Basically, the proof is only the application of the main
correctness theorem to well-chosen arguments.


\asection{Dealing with fields
\tacindex{field}
\tacindex{field\_simplify}
\tacindex{field\_simplify\_eq}}


The {\tt field} tactic is  an extension of the {\tt ring} to deal with
rational expresision. Given a rational expression $F=0$. It first reduces the expression $F$ to a common denominator $N/D= 0$ where $N$ and $D$ are two ring
expressions.
For example, if we take $F = (1 - 1/x) x - x + 1$, this gives 
$ N= (x -1)  x - x^2 + x$ and $D= x$. It then calls {\tt ring} 
to solve $N=0$. Note that {\tt field} also generates non-zero conditions
for all the denominators it encounters in the reduction.
In our example, it generates the condition $x \neq 0$. These
conditions appear as one subgoal which is a conjunction if there are
several denominators.
Non-zero conditions are {\it always} polynomial expressions. For example 
when reducing the expression $1/(1 + 1/x)$, two side conditions are
generated: $x\neq 0$ and $x + 1 \neq 0$. Factorized expressions are
broken since a field is an integral domain, and when the equality test
on coefficients is complete w.r.t. the equality of the target field,
constants can be proven different from zero automatically.

The tactic must be loaded by \texttt{Require Import Field}. New field
structures can be declared to the system with the \texttt{Add Field}
command (see below). The field of real numbers is defined in module
\texttt{RealField} (in texttt{contrib/setoid\_ring}). It is exported
by module \texttt{Rbase}, so that requiring \texttt{Rbase} or
\texttt{Reals} is enough to use the field tactics on real
numbers. Rational numbers in canonical form are also declared as a
field in module \texttt{Qcanon}.


\Example
\begin{coq_eval}
Reset Initial.
\end{coq_eval}
\begin{coq_example}
Require Import Reals.
Open Scope R_scope.
Goal forall x,  x <> 0 ->
   (1 - 1/x) * x - x + 1 = 0.
\end{coq_example}
\begin{coq_example}
intros; field; auto.
\end{coq_example}
\begin{coq_eval}
Abort.
\end{coq_eval}
\begin{coq_example}
Goal forall x y, y <> 0 -> y = x -> x/y = 1.
\end{coq_example}
\begin{coq_example}
intros x y H H1; field [H1]; auto.
\end{coq_example} 
\begin{coq_eval}
Reset Initial.
\end{coq_eval}

\begin{Variants}
  \item {\tt field [\term$_1$ {\ldots} \term$_n$]} decides the equality of two
    terms modulo field operations and rewriting of the equalities
    defined by \term$_1$ {\ldots} \term$_n$. Each of \term$_1$
    {\ldots} \term$_n$ has to be a proof of some equality $m = p$,
    where $m$ is a monomial (after ``abstraction''),
    $p$ a polynomial and $=$ the corresponding equality of the field structure.
    Beware that rewriting works with the equality $m=p$ only if $p$ is a 
    polynomial since rewriting is handled by the underlying {\tt ring}
    tactic.
  \item {\tt field\_simplify} 
     performs the simplification in the conclusion of the goal, $F_1 = F_2$
     becomes $N_1/D_1 = N_2/D_2$. A normalization step (the same as the
     one for rings) is then applied to $N_1$, $D_1$, $N_2$ and
     $D_2$. This way, polynomials remain in factorized form during the
     fraction simplifications. This yields smaller expressions when
     reducing to the same denominator since common factors can be
     cancelled.

  \item {\tt field\_simplify   [\term$_1$ {\ldots} \term$_n$]}
     performs the simplification in the conclusion of the goal using
    the equalities
    defined by \term$_1$ {\ldots} \term$_n$.

  \item {\tt field\_simplify   [\term$_1$ {\ldots} \term$_n$] $t_1$ \ldots
$t_m$}
     performs the simplification in the terms $t_1$ \ldots $t_m$
    of the conclusion of the goal using
    the equalities
    defined by \term$_1$ {\ldots} \term$_n$. 

  \item {\tt field\_simplify in $H$}  
     performs the simplification in the assumption $H$.

  \item {\tt field\_simplify   [\term$_1$ {\ldots} \term$_n$] in $H$}
     performs the simplification in the assumption $H$ using
    the equalities
    defined by \term$_1$ {\ldots} \term$_n$. 

  \item {\tt field\_simplify   [\term$_1$ {\ldots} \term$_n$] $t_1$ \ldots
$t_m$ in $H$}
     performs the simplification in the terms $t_1$ \ldots $t_n$
    of the assumption $H$ using
    the equalities
    defined by \term$_1$ {\ldots} \term$_m$. 

  \item {\tt field\_simplify\_eq}
     performs the simplification in the conclusion of the goal removing
     the denominator. $F_1 = F_2$
     becomes $N_1 D_2 = N_2  D_1$.

  \item {\tt field\_simplify\_eq   [\term$_1$ {\ldots} \term$_n$]}
     performs the simplification in the conclusion of the goal using
    the equalities
    defined by \term$_1$ {\ldots} \term$_n$. 

  \item {\tt field\_simplify\_eq} in $H$
     performs the simplification in the assumption $H$.

  \item {\tt field\_simplify\_eq   [\term$_1$ {\ldots} \term$_n$] in $H$}
     performs the simplification in the assumption $H$ using
    the equalities
    defined by \term$_1$ {\ldots} \term$_n$. 
\end{Variants}

\asection{Adding a new field structure
\comindex{Add Field}}

Declaring a new field consists in proving that a field signature (a
carrier set, an equality, and field operations: {\tt
Field\_theory.field\_theory} and {\tt Field\_theory.semi\_field\_theory})
satisfies the field axioms. Semi-fields (fields without $+$ inverse) are
also supported. The equality can be either Leibniz equality, or any
relation declared as a setoid (see~\ref{setoidtactics}). The definition
of fields and semi-fields is:
\begin{verbatim}
Record field_theory : Prop := mk_field {
    F_R : ring_theory rO rI radd rmul rsub ropp req;
    F_1_neq_0 : ~ 1 == 0;
    Fdiv_def : forall p q, p / q == p * / q;
    Finv_l : forall p, ~ p == 0 ->  / p * p == 1
}.

Record semi_field_theory : Prop := mk_sfield {
    SF_SR : semi_ring_theory rO rI radd rmul req;
    SF_1_neq_0 : ~ 1 == 0;
    SFdiv_def : forall p q, p / q == p * / q;
    SFinv_l : forall p, ~ p == 0 ->  / p * p == 1
}.
\end{verbatim}

The result of the normalization process is a fraction represented by
the following type:
\begin{verbatim}
Record linear : Type := mk_linear {
   num : PExpr C;
   denum : PExpr C;
   condition : list (PExpr C) }.
\end{verbatim}
where {\tt num} and {\tt denum} are the numerator and denominator;
{\tt condition} is a list of expressions that have appeared as a
denominator during the normalization process. These expressions must
be proven different from zero for the correctness of the algorithm.

The syntax for adding a new field is {\tt Add Field $name$ : $field$
($mod_1$,\dots,$mod_2$)}.  The name is not relevent. It is just used
for error messages. $field$ is a proof that the field signature
satisfies the (semi-)field axioms. The optional list of modifiers is
used to tailor the behaviour of the tactic. Since field tactics are
built upon ring tactics, all mofifiers of the {\tt Add Ring}
apply. There is only one specific modifier:
\begin{description}
\item[completeness \term] allows the field tactic to prove
  automatically that the image of non-zero coefficients are mapped to
  non-zero elements of the field. \term is a proof of {\tt forall x y,
    [x] == [y] -> x?=!y = true}, which is the completeness of equality
  on coefficients w.r.t. the field equality.
\end{description}

\asection{Legacy implementation}

\Warning This tactic is the {\tt ring} tactic of previous versions of
\Coq{} and it should be considered as deprecated. It will probably be
removed in future releases. It has been kept only for compatibility
reasons and in order to help moving existing code to the newer
implementation described above. For more details, please refer to the
Coq Reference Manual, version 8.0.


\subsection{\tt legacy ring \term$_1$ \dots\ \term$_n$
\tacindex{legacy ring}
\comindex{Add Legacy Ring}
\comindex{Add Legacy Semi Ring}}

This tactic, written by Samuel Boutin and Patrick Loiseleur, applies
associative commutative rewriting on every ring.  The tactic must be
loaded by \texttt{Require Import LegacyRing}. The ring must be declared in
the \texttt{Add Ring} command. The ring of booleans
is predefined; if one wants to use the tactic on \texttt{nat} one must
first require the module \texttt{LegacyArithRing}; for \texttt{Z}, do
\texttt{Require Import LegacyZArithRing}; for \texttt{N}, do \texttt{Require
Import LegacyNArithRing}.

The terms \term$_1$, \dots, \term$_n$ must be subterms of the goal
conclusion. The tactic \texttt{ring} normalizes these terms
w.r.t. associativity and commutativity and replace them by their
normal form.

\begin{Variants}
\item \texttt{legacy ring} When the goal is an equality $t_1=t_2$, it
  acts like \texttt{ring\_simplify} $t_1$ $t_2$ and then
  solves the equality by reflexivity.

\item \texttt{ring\_nat} is a tactic macro for \texttt{repeat rewrite
    S\_to\_plus\_one; ring}. The theorem \texttt{S\_to\_plus\_one} is a
  proof that \texttt{forall (n:nat), S n = plus (S O) n}.

\end{Variants}

You can have a look at the files \texttt{LegacyRing.v},
\texttt{ArithRing.v}, \texttt{ZArithRing.v} to see examples of the
\texttt{Add Ring} command.

\subsection{Add a ring structure}

It can be done in the \Coq toplevel (No ML file to edit and to link
with \Coq). First, \texttt{ring} can handle two kinds of structure:
rings and semi-rings. Semi-rings are like rings without an opposite to
addition. Their precise specification (in \gallina) can be found in
the file

\begin{quotation}
\begin{verbatim}
contrib/ring/Ring_theory.v
\end{verbatim}
\end{quotation}

The typical example of ring is \texttt{Z}, the typical
example of semi-ring is \texttt{nat}.

The specification of a
ring is divided in two parts: first the record of constants
($\oplus$, $\otimes$, 1, 0, $\ominus$) and then the theorems
(associativity, commutativity, etc.).

\begin{small}
\begin{flushleft}
\begin{verbatim}
Section Theory_of_semi_rings.

Variable A : Type.
Variable Aplus : A -> A -> A.
Variable Amult : A -> A -> A.
Variable Aone : A.
Variable Azero : A.
(* There is also a "weakly decidable" equality on A. That means 
  that if (A_eq x y)=true then x=y but x=y can arise when 
  (A_eq x y)=false. On an abstract ring the function [x,y:A]false
  is a good choice. The proof of A_eq_prop is in this case easy. *)
Variable Aeq : A -> A -> bool.

Record Semi_Ring_Theory : Prop :=
{ SR_plus_sym  : (n,m:A)[| n + m == m + n |];
  SR_plus_assoc : (n,m,p:A)[| n + (m + p) == (n + m) + p |];

  SR_mult_sym : (n,m:A)[| n*m == m*n |];
  SR_mult_assoc : (n,m,p:A)[| n*(m*p) == (n*m)*p |];
  SR_plus_zero_left :(n:A)[| 0 + n == n|];
  SR_mult_one_left : (n:A)[| 1*n == n |];
  SR_mult_zero_left : (n:A)[| 0*n == 0 |];
  SR_distr_left   : (n,m,p:A) [| (n + m)*p == n*p + m*p |];
  SR_plus_reg_left : (n,m,p:A)[| n + m == n + p |] -> m==p;
  SR_eq_prop : (x,y:A) (Is_true (Aeq x y)) -> x==y
}.
\end{verbatim}
\end{flushleft}
\end{small}

\begin{small}
\begin{flushleft}
\begin{verbatim}
Section Theory_of_rings.

Variable A : Type.

Variable Aplus : A -> A -> A.
Variable Amult : A -> A -> A.
Variable Aone : A.
Variable Azero : A.
Variable Aopp : A -> A.
Variable Aeq : A -> A -> bool.


Record Ring_Theory : Prop :=
{ Th_plus_sym  : (n,m:A)[| n + m == m + n |];
  Th_plus_assoc : (n,m,p:A)[| n + (m + p) == (n + m) + p |];
  Th_mult_sym : (n,m:A)[| n*m == m*n |];
  Th_mult_assoc : (n,m,p:A)[| n*(m*p) == (n*m)*p |];
  Th_plus_zero_left :(n:A)[| 0 + n == n|];
  Th_mult_one_left : (n:A)[| 1*n == n |];
  Th_opp_def : (n:A) [| n + (-n) == 0 |];
  Th_distr_left   : (n,m,p:A) [| (n + m)*p == n*p + m*p |];
  Th_eq_prop : (x,y:A) (Is_true (Aeq x y)) -> x==y
}.
\end{verbatim}
\end{flushleft}
\end{small}

To define a ring structure on A, you must provide an addition, a
multiplication, an opposite function and two unities 0 and 1.

You must then prove all theorems that make
(A,Aplus,Amult,Aone,Azero,Aeq) 
a ring structure, and pack them with the \verb|Build_Ring_Theory| 
constructor.

Finally to register a ring the syntax is:

\comindex{Add Legacy Ring}
\begin{quotation}
  \texttt{Add Legacy Ring} \textit{A Aplus Amult Aone Azero Ainv Aeq T}
  \texttt{[} \textit{c1 \dots cn} \texttt{].}
\end{quotation}

\noindent where \textit{A} is a term of type \texttt{Set}, 
\textit{Aplus} is a term of type \texttt{A->A->A},
\textit{Amult} is a term of type \texttt{A->A->A},
\textit{Aone} is a term of type \texttt{A},
\textit{Azero} is a term of type \texttt{A},
\textit{Ainv} is a term of type \texttt{A->A},
\textit{Aeq} is a term of type \texttt{A->bool},
\textit{T} is a term of type 
\texttt{(Ring\_Theory }\textit{A Aplus Amult Aone Azero Ainv
  Aeq}\texttt{)}.
The arguments \textit{c1 \dots cn}, 
are the names of constructors which define closed terms: a
subterm will be considered as a constant if it is either one of the
terms \textit{c1 \dots cn} or the application of one of these terms to
closed terms. For \texttt{nat}, the given constructors are \texttt{S}
and \texttt{O}, and the closed terms are \texttt{O}, \texttt{(S O)},
\texttt{(S (S O))}, \ldots

\begin{Variants}
\item \texttt{Add Legacy Semi Ring} \textit{A Aplus Amult Aone Azero Aeq T} 
  \texttt{[} \textit{c1 \dots\ cn} \texttt{].}\comindex{Add Legacy Semi
    Ring}
  
  There are two differences with the \texttt{Add Ring} command: there
  is no inverse function and the term $T$ must be of type
  \texttt{(Semi\_Ring\_Theory }\textit{A Aplus Amult Aone Azero
    Aeq}\texttt{)}.

\item \texttt{Add Legacy Abstract Ring} \textit{A Aplus Amult Aone Azero Ainv 
    Aeq T}\texttt{.}\comindex{Add Legacy Abstract Ring}
  
  This command should be used for when the operations of rings are not
  computable; for example the real numbers of
  \texttt{theories/REALS/}. Here $0+1$ is not beta-reduced to $1$ but
  you still may want to \textit{rewrite} it to $1$ using the ring
  axioms. The argument \texttt{Aeq} is not used; a good choice for
  that function is \verb+[x:A]false+.

\item \texttt{Add Legacy Abstract Semi Ring} \textit{A Aplus Amult Aone Azero
    Aeq T}\texttt{.}\comindex{Add Legacy Abstract Semi Ring}

\end{Variants}

\begin{ErrMsgs}
\item \errindex{Not a valid (semi)ring theory}.
 
  That happens when the typing condition does not hold.
\end{ErrMsgs}

Currently, the hypothesis is made than no more than one ring structure
may be declared for a given type in \texttt{Set} or \texttt{Type}.
This allows automatic detection of the theory used to achieve the
normalization. On popular demand, we can change that and allow several
ring structures on the same set.

The table of ring theories is compatible with the \Coq\ 
sectioning mechanism. If you declare a ring inside a section, the
declaration will be thrown away when closing the section.
And when you load a compiled file, all the \texttt{Add Ring}
commands of this file that are not inside a section will be loaded.

The typical example of ring is \texttt{Z}, and the typical example of
semi-ring is \texttt{nat}. Another ring structure is defined on the
booleans. 

\Warning Only the ring of booleans is loaded by default with the
\texttt{Ring} module. To load the ring structure for \texttt{nat},
load the module \texttt{ArithRing}, and for \texttt{Z},
load the module \texttt{ZArithRing}.

\subsection{\tt legacy field
\tacindex{legacy field}}

This tactic written by David~Delahaye and Micaela~Mayero solves equalities
using commutative field theory. Denominators have to be non equal to zero and,
as this is not decidable in general, this tactic may generate side conditions
requiring some expressions to be non equal to zero. This tactic must be loaded
by {\tt Require Import LegacyField}. Field theories are declared (as for
{\tt legacy ring}) with
the {\tt Add Legacy Field} command.

\subsection{\tt Add Legacy Field
\comindex{Add Legacy Field}}

This vernacular command adds a commutative field theory to the database for the
tactic {\tt field}. You must provide this theory as follows:
\begin{flushleft}
{\tt Add Legacy Field {\it A} {\it Aplus} {\it Amult} {\it Aone} {\it Azero} {\it
Aopp} {\it Aeq} {\it Ainv} {\it Rth} {\it Tinvl}}
\end{flushleft}
where {\tt {\it A}} is a term of type {\tt Type}, {\tt {\it Aplus}} is
a term of type {\tt A->A->A}, {\tt {\it Amult}} is a term of type {\tt
  A->A->A}, {\tt {\it Aone}} is a term of type {\tt A}, {\tt {\it
    Azero}} is a term of type {\tt A}, {\tt {\it Aopp}} is a term of
type {\tt A->A}, {\tt {\it Aeq}} is a term of type {\tt A->bool}, {\tt
  {\it Ainv}} is a term of type {\tt A->A}, {\tt {\it Rth}} is a term
of type {\tt (Ring\_Theory {\it A Aplus Amult Aone Azero Ainv Aeq})},
and {\tt {\it Tinvl}} is a term of type {\tt forall n:{\it A},
  {\~{}}(n={\it Azero})->({\it Amult} ({\it Ainv} n) n)={\it Aone}}.
To build a ring theory, refer to Chapter~\ref{ring} for more details.

This command adds also an entry in the ring theory table if this theory is not
already declared. So, it is useless to keep, for a given type, the {\tt Add
Ring} command if you declare a theory with {\tt Add Field}, except if you plan
to use specific features of {\tt ring} (see Chapter~\ref{ring}). However, the
module {\tt ring} is not loaded by {\tt Add Field} and you have to make a {\tt
Require Import Ring} if you want to call the {\tt ring} tactic.

\begin{Variants}

\item {\tt Add Legacy Field {\it A} {\it Aplus} {\it Amult} {\it Aone} {\it Azero}
{\it Aopp} {\it Aeq} {\it Ainv} {\it Rth} {\it Tinvl}}\\
{\tt \phantom{Add Field }with minus:={\it Aminus}}

Adds also the term {\it Aminus} which must be a constant expressed by
means of {\it Aopp}.

\item {\tt Add Legacy Field {\it A} {\it Aplus} {\it Amult} {\it Aone} {\it Azero}
{\it Aopp} {\it Aeq} {\it Ainv} {\it Rth} {\it Tinvl}}\\
{\tt \phantom{Add Legacy Field }with div:={\it Adiv}}

Adds also the term {\it Adiv} which must be a constant expressed by
means of {\it Ainv}.

\end{Variants}

\SeeAlso \cite{DelMay01} for more details regarding the implementation of {\tt
legacy field}.

\asection{History of \texttt{ring}}

First Samuel Boutin designed the tactic \texttt{ACDSimpl}. 
This tactic did lot of rewriting. But the proofs
terms generated by rewriting were too big for \Coq's type-checker.
Let us see why:

\begin{coq_eval}
Require Import ZArith.
Open Scope Z_scope.
\end{coq_eval}
\begin{coq_example}
Goal forall x y z:Z, x + 3 + y + y * z = x + 3 + y + z * y.
\end{coq_example}
\begin{coq_example*}
intros; rewrite (Zmult_comm y z); reflexivity.
Save toto.
\end{coq_example*}
\begin{coq_example}
Print  toto.
\end{coq_example}

At each step of rewriting, the whole context is duplicated in the proof
term. Then, a tactic that does hundreds of rewriting generates huge proof
terms. Since \texttt{ACDSimpl} was too slow, Samuel Boutin rewrote it
using reflection (see his article in TACS'97 \cite{Bou97}). Later, the
stuff was rewritten by Patrick
Loiseleur: the new tactic does not any more require \texttt{ACDSimpl}
to compile and it makes use of $\beta\delta\iota$-reduction 
not only to replace the rewriting steps, but also to achieve the
interleaving of computation and 
reasoning (see \ref{DiscussReflection}). He also wrote a
few ML code for the \texttt{Add Ring} command, that allow to register
new rings dynamically.

Proofs terms generated by \texttt{ring} are quite small, they are
linear in the number of $\oplus$ and $\otimes$ operations in the
normalized terms. Type-checking those terms requires some time because it
makes a large use of the conversion rule, but
memory requirements are much smaller. 

\asection{Discussion}
\label{DiscussReflection}

Efficiency is not the only motivation to use reflection
here. \texttt{ring} also deals with constants, it rewrites for example the
expression $34 + 2*x -x + 12$ to the expected result $x + 46$. For the
tactic \texttt{ACDSimpl}, the only constants were 0 and 1. So the
expression $34 + 2*(x - 1) + 12$ is interpreted as 
$V_0 \oplus V_1 \otimes (V_2 \ominus 1) \oplus V_3$, 
with the variables mapping 
$\{V_0 \mt 34; V_1 \mt 2; V_2 \mt x; V_3 \mt 12 \}$. Then it is
rewritten to $34 - x + 2*x + 12$, very far from the expected
result. Here rewriting is not sufficient: you have to do some kind of
reduction (some kind of \textit{computation}) to achieve the
normalization.

The tactic \texttt{ring} is not only faster than a classical one:
using reflection, we get for free integration of computation and
reasoning that would be very complex to implement in the classic fashion.

Is it the ultimate way to write tactics?  The answer is: yes and
no. The \texttt{ring} tactic uses intensively the conversion rule of
\CIC, that is replaces proof by computation the most as it is
possible. It can be useful in all situations where a classical tactic
generates huge proof terms. Symbolic Processing and Tautologies are in
that case. But there are also tactics like \texttt{auto} or
\texttt{linear} that do many complex computations, using side-effects
and backtracking, and generate a small proof term. Clearly, it would
be significantly less efficient to replace them by tactics using
reflection.

Another idea suggested by Benjamin Werner: reflection could be used to
couple an external tool (a rewriting program or a model checker) with
\Coq. We define (in \Coq) a type of terms, a type of \emph{traces},
and prove a correction theorem that states that \emph{replaying
traces} is safe w.r.t some interpretation. Then we let the external
tool do every computation (using side-effects, backtracking,
exception, or others features that are not available in pure lambda
calculus) to produce the trace: now we can check in Coq{} that the
trace has the expected semantic by applying the correction lemma.
 
%%% Local Variables: 
%%% mode: latex
%%% TeX-master: "Reference-Manual"
%%% End: 
